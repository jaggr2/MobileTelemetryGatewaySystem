\chapter{Aufgabenstellung}
\label{chap:aufgabenstellung}

Von Seiten Bern Formula Student wurden folgende Zielvorstellungen an die Telemetrie formuliert:

\begin{itemize}
\itemsep 1pt \parskip 0pt \parsep 0pt
\item Übermitteln möglichst aller Daten, die auf den drei CAN-Bus-Systemen des Fahrzeuges übertragen werden. Übertragene Informationen sind beispielsweise Beschleunigung, Bremsstärke, Lenkeinschlag, Geschwindig, diverse Temperaturwerte, Drehzahlen der Motoren, etc.

\item Erfassen von zusätzlichen Temperaturwerten wie beispielsweise Kühlluft-Eingangs sowie Ausgangstemperatur. Dafür werden kontaktlose Temperatursensoren benötigt, die auf einen schwarz markierten Referenzpunkt gerichtet werden.

\end{itemize}

Erweitert wurde die Aufgabenstellung von Andreas Danuser durch folgende Konstruktionsvorgaben:

\begin{itemize}
\itemsep 1pt \parskip 0pt \parsep 0pt
\item Die einzelnen Komponenten der Telemterie sollen möglichst universell , sprich auch in einem komplett anderen Einsatzzweck, verwendbar sein.
\item Das System soll modular aufgebaut werden.

\end{itemize}
