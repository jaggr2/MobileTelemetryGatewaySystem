\chapter{Aufgabenstellung}
\label{chap:aufgabenstellung}

Von Seiten Bern Formula Student wurde folgende Zielvorstellung an die Telemetrie formuliert:

\textit{Grundsätzliches möglichst alle Daten übermitteln, die auf den drei CAN-Bus-Systemen des Fahrzeuges übertragen werden.} 

Zum jetzigen Zeitpunkt sind dies:

\begin{itemize}
	\itemsep 1pt \parskip 0pt \parsep 0pt
	\item \textbf{Eingangsgrössen Fahrer}		
		\begin{itemize}
			\itemsep 1pt \parskip 0pt \parsep 0pt
			\item Fahrpedalstellung		\textit{CAN 1}
			\item Bremspedalstellung	\textit{CAN 1}
			\item Bremsdruck			\textit{CAN 1}
			\item Lenkwinkel			\textit{CAN 1}
		\end{itemize}
	\item \textbf{Motor und Kühlsystem}		
		\begin{itemize}
			\itemsep 1pt \parskip 0pt \parsep 0pt
			\item Soll-Moment			\textit{CAN 2/3}
			\item Ist-Moment			\textit{CAN 2/3}
			\item Soll-Drehzahl			\textit{CAN 2/3}
			\item Ist-Drehzahl			\textit{CAN 2/3}
			\item Motortemperatur		\textit{CAN 2/3}
			\item Kühlmitteltemperatur	\textit{CAN 1}
		\end{itemize}		
	\item \textbf{Batterie}		
		\begin{itemize}
			\itemsep 1pt \parskip 0pt \parsep 0pt
			\item HV-Spannung				\textit{CAN 1}
			\item HV-Stromg					\textit{CAN 1}
			\item HV-Batterie Ladestand		\textit{CAN 1}
			\item GLV Batterie Spannung		\textit{Wireless über Sming}
		\end{itemize}
					
	\item \textbf{Fahrzeug}		
	\begin{itemize}
		\itemsep 1pt \parskip 0pt \parsep 0pt
		\item Geschwindigkeit						\textit{CAN 1}
		\item Beschleunigung						\textit{CAN 1}
		\item Drehrate 								\textit{CAN 1}
		\item Reifentemperatur						\textit{Wireless über Sming}
		\item Bremstemperatur						\textit{Wireless über Sming}
		\item Bodentemperatur						\textit{Wireless über Sming}
		\item Lufttempertur bei Kühlereintritt		\textit{Wireless über Sming}
		\item Lufttempertur bei Kühleraustritt		\textit{Wireless über Sming}
		\item Dämpferweg (optional)					\textit{Wireless über Sming}
		
	\end{itemize}			
\end{itemize}

Die Anbindung "Wireless über Sming" bedeutet, dass wir selber Sensoren zur Erfassung der gewünschten Messgrössen liefern sollen. Die Sensoren sollen dabei drahtlos über den Bluetooth-Stack des Smings kommunizieren.

Erweitert wurde die Aufgabenstellung von Andreas Danuser durch die Konstruktionsvorgabe, dass die einzelnen Komponenten der Telemetrie möglichst universell, sprich als eigenständiges Modul auch in einem komplett anderen Einsatzzweck, verwendbar sein sollen.

